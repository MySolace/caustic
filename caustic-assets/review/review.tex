\documentclass[journal]{../styles/IEEEtran}

\usepackage{amsmath}
\usepackage{amssymb}
\usepackage{amsthm}

\begin{document}

% Title
\title{Caustic: Literature Review}
\author{Ashwin Madavan}
\date{15 September 2017}
\maketitle

% HyperDex
\section{HyperDex}

  \subsection{Hyperspace Hashing}
  Hyperspace Hashing colocates regions of the n-dimensional hyperspace induced by the attributes of 
  an object in order to perform efficient secondary indexing. Each column of a table corresponds to 
  a different dimension in a hyperspace. Each physical server is responsible for storing data within 
  a particular region in the hyperspace. By partitioning data in this manner, you can optimize 
  searches (which correspond to hyperplanes) by reducing the number of server that they need to 
  contact.

  \subsection{Value Dependent Chaining}
  Object updates occur in two linear passes over all servers containing an object. In the first 
  pass, each server creates a pending update and propagates the change to the next server. In the 
  second pass, each server commits the pending update. Linear object updates means that updates to 
  an object have to *sequentially* pass through every server that it is physically stored on. While 
  this commit protocol ensures consistency, it will not be as performant for larger objects.

  \subsection{Performance}
  I don't understand how the performance of their system is so significantly better than Cassandra. 
  I'm always skeptical of benchmarks, particularly when they show such significant improvements over 
  high-performance, enterprise-grade solutions. Something about these benchmarks just feels wrong. 
  Both HyperDex and Cassandra use a log-based storage engine (HyperDex stores a log file for each 
  object, and Cassandra uses LSM trees), so it probably isn't the on-disk representation that causes 
  the difference in performance.

  A closer look at the benchmarks reveals that Cassandra actually performs better on writes than 
  HyperDex (which is unsurprising given that HyperDex uses a linear commit protocol) and that the 
  higher read performance of HyperDex causes it benchmark better. Why might HyperDex have higher 
  read performance? In Cassandra, key-value pairs are replicated $f$ times, and, by default, 
  Cassandra returns the result of a read on one of the replicas of a key. In HyperDex, objects are 
  replicated $f$ times across $n$ subspaces. Therefore, there are $n$ times as many replicas of an 
  object as there are in Cassandra. Cassandra read throughput scales linearly with the number of 
  nodes. In the benchmark, objects in HyperDex are replicated to three subspaces (key subspace + two 
  attribute subspaces). If Cassandra had three times as many replicas, it would have three times the 
  read throughput. Clearly, the benchmarks seem biased toward HyperDex, because they are replicating 
  objects many more times than Cassandra.

% Yesquel
\section{Yesquel}

  \subsection{Decoupling Query Processor and Storage Engines}
  One of the main achievements of Yesquel is that it shows that dropping the SQL front end and 
  compromising consistency are not fundamentally necessary to achieve performance at scale. The
  paper acknowledges that, 
  
    \begin{quote}
    Traditionally, the storage system had been a SQL database system, which is convenient for the
    developer. However, with the emergence of large Web applications in the past decade, the SQL
    database became a scalability bottleneck. Developers reacted in two ways. They used application
    specific techniques to improve scalability - which is a laborious and complicates application
    code. Or they replaced database systems with specialized NoSQL systems that scaled well by
    shedding functionality.
    \end{quote}

  However, by decoupling the query processor from the underlying storage engine, Yesquel was able to
  preserve a relational interface over a denormalized key-value store without sacrificing
  performance. I'm not convinced of the novelty of this, because it has been demonstrated before in
  a number of projects (Cassandra, CockroachDB, Aerospike, ArangoDB, etc.). Nonetheless, it is an
  important feature that deserves mention.  

  \subsection{YDBT}
  Perhaps the most novel aspect of Yesquel is the Yesquel Distributed Balanced Tree (YDBT). My
  understanding is that it is functionally similar to a B+ tree, except that its nodes are split
  both when they grow too large and when they are under too high load.

  \subsection{Integrated Caching}
  This feature has already been implemented in Caustic, and it is reassuring that other people are
  doing the same. The idea is that transactions can cache incoherently without compromising data
  integrity, by speculating about the value of a key from a potentially stale version in cache and
  then validating these cached version on commit.
  
  \subsection{Performance}
  Benchmarks show that the system performs in the ballpark of Redis. While it is interesting that 
  their YDBTs perform similarly to hash-table systems, I'm still not convinced that they provide any
  additional functionality that cannot be directly encoded into a hash-table.

\end{document}
